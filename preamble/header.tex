%---------------------------------
% সর্বজনীন হচ্ছে একটা বিশেষণ যার অর্থ  সবার মধ্যে প্রবীণ ও জ্যেষ্ঠ`. এখানে যে বানানটি হবে সেটা হলো সর্বজনীন, যার অর্থ সর্বসাধারণের জন্য অনুষ্ঠিত,  সকলের জন্য উদ্দিষ্ট.
%
%উদাহরণ: ক্যামেরুনের  বর্তমান রাষ্ট্রপতি পল বোয়া সার্বজনীন নেতা। (এখানে সার্বজনীন ব্যবহার হবে কারণ ওনার থেকে অধিক বয়স্ক নেতা বিশ্বে আর কেউ নেই)
%
%উদাহরণ: মানবাধিকার সর্বজনীন স্বীকৃত অধিকার।  নয়াবসান সর্বজনীন দুর্গোৎসব পূজা কমিটি।
%
%এখানে সঠিক বানানটি হবে: সর্বজনীন
%----------------------------------------------
\newcommand{\myheader}{ \BgThispage \copyrightnotice
	\scalebox{3}[3]{ নয়াবসান সর্বজনীন দুর্গোৎসব পূজা কমিটি}\large
	\begin{tcolorbox}[colback=green!5!white,colframe=green!75!black,width=0.5\textwidth,left=3pt,right=3pt,boxsep=0pt ]
		\large সংস্থাপিতঃ ১৪০৯ $\bigwhitestar$ ২১তম বর্ষ $\bigwhitestar$ সনঃ ১৪২৯ (ইংঃ ২০২২)   
	\end{tcolorbox}
	গ্রামঃ নয়াবসান $\bigwhitestar$ পোস্ট+থানাঃ গোপীবল্লভপুর $\bigwhitestar$ জেলাঃ ঝাড়গ্রাম $\bigwhitestar$ পিনঃ ৭২১৫০৬
}
